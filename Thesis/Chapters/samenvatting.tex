\begin{abstract*}
\english{Adversarial} aanvallen creëren kleine perturbaties die aan afbeeldingen kunnen worden toegevoegd. Door deze toevoeging, worden deze afbeeldingen niet langer juist geclassificeerd door een kunstmatige intelligentie klasseerder. Moderne verdedigingsmechanismen proberen om zo een aanval te deteteren op basis van gelijkenissen tussen opeenvolgend verstuurde afbeeldingen. Dit werk implementeert een nieuwe \english{adversarial} aanval genaamd PSO-BBA. Deze aanval is genoemd naar de twee onderdelen, namelijk \english{Particle Swarm Optimization} en \english{Biased Boundary Attack} \cite{brunner_guessing_2019}. Het doel van de nieuwe aanval is het omzeilen van het verdediginsmechanisme. Verschillende verbeteringen worden voorgesteld en aangebracht aan het algoritme. Enkele van deze verbeteringen zijn: een gedistribueerde manier voor het versturen van afbeeldingen over meerdere machines en het groeperen van aanvallen. De uiteindelijke aanval wordt minder gedetecteerd dan de modernste aanvallen, maar hij is minder performant. De aanvaller moet dus een afweging maken afhankelijk van het belangrijkste doel, detectie omzeilen of hoge performantie.
\end{abstract*}
