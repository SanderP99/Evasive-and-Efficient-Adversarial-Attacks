\begin{abstract*}
%\english{Adversarial} aanvallen creëren kleine perturbaties die aan afbeeldingen kunnen worden toegevoegd. Door deze toevoeging, worden deze afbeeldingen niet langer juist geclassificeerd door een kunstmatige intelligentie klasseerder. Moderne verdedigingsmechanismen proberen om zo een aanval te deteteren op basis van gelijkenissen tussen opeenvolgend verstuurde afbeeldingen. Dit werk implementeert een nieuwe \english{adversarial} aanval genaamd PSO-BBA. Deze aanval is genoemd naar de twee onderdelen, namelijk \english{Particle Swarm Optimization} en \english{Biased Boundary Attack} \cite{brunner_guessing_2019}. Het doel van de nieuwe aanval is het omzeilen van het verdediginsmechanisme. Verschillende verbeteringen worden voorgesteld en aangebracht aan het algoritme. Enkele van deze verbeteringen zijn: een gedistribueerde manier voor het versturen van afbeeldingen over meerdere machines en het groeperen van aanvallen. De uiteindelijke aanval wordt minder gedetecteerd dan de modernste aanvallen, maar hij is minder performant. De aanvaller moet dus een afweging maken afhankelijk van het belangrijkste doel, detectie omzeilen of hoge performantie.
Afbeelding classificatie modellen hebben last van \english{adversarial} afbeeldingen. Deze zijn schijnbaar identiek aan elkaar, maar worden door het model toch verschillend ingedeeld. De algoritmes die gebruikt worden om dergelijke \english{adversarial} afbeeldingen te maken, worden \english{adversarial} aanvallen genoemd. Moderne verdedigingsmechanismen proberen om aanvallen te detecteren op basis van gelijkenissen tussen opeenvolgend verstuurde afbeeldingen. Dit mechanisme maakt de veronderstelling dat het mogelijk is om de bron van elke afbeelding te traceren en dat verschillende gebruikers van het model niet samenwerken. Dit werk buit deze veronderstellingen uit. Dit wordt gedaan door het implementeren van een nieuwe familie van \english{adversarial} aanvallen. Deze familie verstuurt afbeeldingen naar het model op een gedistribueerde manier, hetgeen maakt dat het verdedigingsmechanisme denkt dat ze van verschillende gebruikers komen. Deze aanval heet PSO-BBA en is vernoemd naar de twee onderdelen, namelijk \english{Particle Swarm Optimization} en \english{Biased Boundary Attack} \cite{brunner_guessing_2019}. Het grote voordeel van PSO zijn de meerdere startposities waarvan de aanval gebruik kan maken. Verschillende distributie technieken zijn onderzocht, maar geen van alle bleek duidelijk beter dan de rest. Om de aanval minder detecteerbaar te maken worden tussentijds andere afbeeldingen, die los van de aanval staan, naar het model verstuurd. Dit ging echter ten koste van de performantie van de aanval. Het tussentijds versturen van andere afbeeldingen die wel deel uit maken van een aanval blijkt wel een goede oplossing om het aantal detecties te verlagen zonder de performantie te schaden. De uiteindelijke aanval wordt minder gedetecteerd dan de modernste aanvallen, maar hij is minder performant. Het is daarom nog steeds een goede kandidaat als de kost geassocieerd met een detectie heel hoog is.
\end{abstract*}
