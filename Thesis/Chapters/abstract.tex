\begin{abstract}
% Adversarial examples
% Defense
% New attack
% More evasive
%Adversarial attacks craft small perturbations that can be added to inputs, so that they are classified incorrectly by a classifier. Current defensive schemes try to flag these attacks based on the similarity between successive submitted queries. This work aims to implement a new adversarial attack that is able to bypass this detection mechanism. This attack is called PSO-BBA after the two constituent algorithms Particle Swarm Optimization and Biased Boundary Attack \cite{brunner_guessing_2019}. Other improvements, such as distributing the query submission over multiple nodes or grouping attacks together have been explored as well. The final attack is more evasive than current state-of-the-art attacks, but is slightly less efficient. The attacker has to make a trade-off depending on the primary goal of the attack.
Image classification models are prone to adversarial examples. Adversarial examples are perceptually identical to original inputs, but they are classified differently. The examples are crafted using adversarial attack algorithms. Current defensive schemes try to flag these attacks based on the similarity between successive submitted queries. The scheme assumes that it can trace back the origin of every query to a user and that there is no cooperation possible between users. This works aims to exploit these assumptions by implementing a new family of adversarial attacks that uses distributed query submission. This tricks the defensive scheme into thinking that the queries originate from different users. The attack is called PSO-BBA after the two constituent algorithms Particle Swarm Optimization and Biased Boundary Attack \cite{brunner_guessing_2019}. The main advantage of PSO are the different starting points for the algorithm. Different distribution schemes were explored, but none of them proved to be a significant winner. In order to increase the evasiveness, the insertion of other (benign) queries was considered, but this approach proved to hamper the efficiency too much. The insertion of queries originating from other concurrent attacks did improve the evasiveness without an added efficiency cost. The final attack is more evasive than current state-of-the-art attacks, but is slightly less efficient, making it a viable candidate if the cost of detection is high.
\end{abstract}