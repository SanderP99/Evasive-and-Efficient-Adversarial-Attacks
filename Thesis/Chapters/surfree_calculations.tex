\chapter{SurFree calculations}\label{app:surfree_calculations}
The following calculations will show how equation \ref{eq:surfree_adversarial} was derived. Assume that the normal vector $\mathbf{n}$ is pointing outside, i.e. to the adversarial region $\mathcal{A}$. A candidate point $\mathbf{z} \in \mathcal{P}$ is adversarial if $(\mathbf{z}-x_b)^\top \mathbf{n} \geq 0$. Under the assumption of a (local) flat decision boundary, this means that both vector $n$ and vector $(\mathbf{z}-x_b)$ are pointing outside. This assumption is justified according to \cite{straight_boundaries}. Below follows the derivation of equation \ref{eq:surfree_adversarial}:\\
\begin{align*}
&(\mathbf{z}-x_b)^\top \mathbf{n} \geq 0 \\
&\Leftrightarrow \left[ d(1-\alpha)(\cos(\theta)\mathbf{u} + \sin(\theta)\mathbf{v}) + x_o - x_b)\right]^\top \mathbf{n} &\geq 0 \annot{Equation \ref{eq:surfree_polar}}\\
&\Leftrightarrow \left[ d(1-\alpha)(\cos(\theta)\mathbf{u} + \sin(\theta)\mathbf{v}) - d\mathbf{u})\right]^\top \mathbf{n} &\geq 0 \annot{$\mathbf{u} = (x_b - x_o) / d$}\\
&\Leftrightarrow [((1-\alpha)\cos(\theta) - 1)\mathbf{u} + ((1-\alpha)\sin(\theta))\mathbf{v}]^\top \mathbf{n} &\geq 0\\
&\Leftrightarrow ((1-\alpha)\cos(\theta) - 1)\cos(\psi) + ((1-\alpha)\sin(\theta))\sin(\psi) &\geq 0 \annot{Equation \ref{eq:surfree_n}}\\
&\Leftrightarrow ((1-\alpha)\cos(\theta) - 1)\cos(\psi) \geq -((1-\alpha)\sin(\theta))\sin(\psi)\\
&\Leftrightarrow (1 - (1-\alpha)\cos(\theta))\cos(\psi) \leq ((1-\alpha)\sin(\theta))\sin(\psi) \annot{$x \geq y \Leftrightarrow -x \leq -y$}\\
&\Leftrightarrow \left| \frac{1 - (1-\alpha)\cos(\theta)}{(1-\alpha)\sin(\theta)}\right| \leq \tan(\psi)\sgn(\theta) \annot{$\tan(x) = \frac{\sin(x)}{\cos(x)}$}
\end{align*}

The absolute value and sign operator are present due to the fact that the inequality should reverse in case of a division by a negative number. Since $\psi$ is restricted to $[-\frac{\pi}{2},\frac{\pi}{2}]$, the cosine of $\psi$ is always positive. Therefore this correction is only needed because of $\sin(\theta)$.\\

The derivative of equation \ref{eq:surfree_adversarial} is derived as follows:
\begin{align*}
\frac{d}{d\theta^*(\alpha)}\frac{1 - (1-\alpha)\cos(\theta)}{(1-\alpha)\sin(\theta)} &=\frac{1 - (1-\alpha)\cos(\arccos(1-\alpha))}{(1-\alpha)\sin(\arccos(1-\alpha))} \annot{$\theta = \pm \arccos(1-\alpha)$} \\
&=\frac{1 - (1-\alpha)(1-\alpha)}{(1-\alpha)\sin(\arccos(1-\alpha))} \annot{$\cos(\arccos(x)) = x$}\\
&=\frac{1 - (1-\alpha)(1-\alpha)}{(1-\alpha)\sqrt{1-(1-\alpha)^2}} \annot{$\sin(\arccos(x)) = \sqrt{1-x^2}$}  \\
&= \frac{\sqrt{1-(1-\alpha)^2}(1-(1-\alpha)^2)}{(1-\alpha)(1-(1-\alpha)^2))}\\
&= \frac{\sqrt{1 - (1-\alpha)^2}}{1-\alpha}\\
&= |\tan(\theta^*(\alpha))|
\end{align*}

The final step is based on the Pythagorean theorem. The tangent of $\theta^*$ is the equal to the fraction on the line before. This is shown visually in Figure \ref{fig:pythagoras}.\\

\begin{figure}[h]
\centering
\begin{tikzpicture}[
d/.style = {draw, fill=teal!30,
                   angle radius=7mm, 
                   angle eccentricity=1.1, 
                   right, inner sep=1pt,
                   font=\footnotesize} 
                   ]
\draw   (0,0) coordinate[] (a) -- node[midway, below]{$x$}
        (5,0) coordinate[] (c) -- node[midway, right]{$\sqrt{1-x^2}$}
        (5,3) coordinate[] (b) -- node[midway, above]{$1$} cycle pic[d] {angle = c--a--b};
\path node[label = {[label distance=5mm]3:$\theta^*$}] (t) at (a) {};
\end{tikzpicture}
\caption[Visual guide to a step in the derivation.]{Visual guide to the final step of the derivation of the derivative of equation \ref{eq:surfree_adversarial}. The tangent of the teal angle is equal to the length of the opposite right angled side divided by the adjacent right angled side.}
\label{fig:pythagoras}
\end{figure}