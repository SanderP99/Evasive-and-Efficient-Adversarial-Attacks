\documentclass[master=cws,masteroption=ai,fleqn,english]{kulemt}
%
% Indien je UTF-8 karakters gebruikt, verwijder "% " uit de volgende lijn
\setup{inputenc=utf8,
% Vul de titel van jouw masterproef hieronder in tussen { en }.
title={Distributed Adversarial Attacks},
%
% Vul hieronder namen in, steeds Voornaam Naam.
% Indien meerdere auteurs, assessoren, assistenten, scheidt hun namen met "\and ".
author={Sander Prenen},
promotor={Prof. dr. ir. W. Joosen \and Dr. ir. D. Preuveneers},
assessor={Ir.\,W. Eetveel\and W. Eetrest},
assistant={V. Rimmer \and I. Tsingenopoulos}
}


% Kies de fonts voor de gewone tekst, bv. Latin Modern
\setup{font=lm}

% Hier kun je dan nog andere pakketten laden of eigen definities voorzien
\usepackage{amsmath, amssymb}
\setlength{\mathindent}{1cm}
\setlength{\parindent}{0mm}
\usepackage{tikz}
\usetikzlibrary{positioning}

\usepackage{pgfplots}
\pgfplotsset{compat=1.16}
%\usepgfplotslibrary{external}
%\tikzexternalize

\usepackage[acronym]{glossaries}
\makenoidxglossaries
\newacronym{ai}{AI}{Artificial Intelligence}
\newacronym{ba}{BA}{Boundary Attack}
\newacronym{bba}{BBA}{Biased Boundary Attack}
\newacronym{relu}{ReLU}{Rectified Linear Unit}
\newacronym{ann}{ANN}{Artificial Neural Network}

% Tenslotte wordt hyperref gebruikt voor pdf bestanden.
% Dit mag verwijderd worden voor de af te drukken versie.
\usepackage[pdfusetitle,colorlinks,plainpages=false]{hyperref}

%
%%%%%%%%% Wijzig niets onder deze regel %%%%%%%%

\begin{document}

\begin{preface}
% CROWN JEWEL, BUSY AS A BEAVER, BLOOD SWEAT AND TEARS, RECHARGE ONE'S BATTERIES, ELVIS HAS LEFT THE BUILDING, FAT LADY SINGING
In front of you lies the thesis: "Efficient and Evasive Distributed Adversarial Attacks using Particle Swarm Optimization". The text is the result of a year of hard work and I am very proud of the final result.\\

The subject seemed very interesting when I first read about it in May of 2021. Having worked on it for the past year, I can guarantee you that it did not only seem an interesting topic, it proved to be one as well. I hope you agree with me once you have read this text and I wish you a lot of enjoyment in reading it.\\
 
This work will be the crown jewel of my academic career. After five years of blood, sweat and tears, albeit not literally, I can hear the fat lady singing in the distance. Before it is completely over, I want to thank everyone that helped me over the course of these five years. I want to thank my family and girlfriend for supporting me on this journey and proofreading all reports I have written, even though the contents of some of them was very difficult to grasp. I want to thank my friends for the fun times and the company during the lessons, some of which seemed to last for days. I want to thank the assessors and members of the jury for taking their time to read this text and triggering my brain one last time with their interesting remarks and questions. Last but not least, I want to thank my supervisor, co-supervisor and assistant-supervisors for the guidance throughout the year and their feedback on my intermediate work.\\

Elvis has left the building!\\
\end{preface}
\begin{abstract}
% Adversarial examples
% Defense
% New attack
% More evasive
Adversarial attacks craft small perturbations that can be added to inputs, so that they are classified incorrectly by a classifier. Current defensive schemes try to flag these attacks based on the similarity between successive submitted queries. This work aims to implement a new adversarial attack that is able to bypass this detection mechanism. This attack is called PSO-BBA after the two constituent algorithms Particle Swarm Optimization and Biased Boundary Attack \cite{brunner_guessing_2019}. Other improvements, such as distributing the query submission over multiple nodes or grouping attacks together have been explored as well. The final attack is more evasive than current state-of-the-art attacks, but is slightly less efficient. The attacker has to make a trade-off depending on the primary goal of the attack.
\end{abstract}
\printnoidxglossary[type=\acronymtype, title=List of Abbreviations]

\tableofcontents
\listoffigures
\listoftables

\mainmatter
\chapter{Introduction}
% General classification (daunting)
% Neural networks
% DNN
% Adversarial attacks
% Defenses
% Bypass
% Overview
One of the most basic tasks in \gls{ai} or more specifically \gls{ml} is the classification of data. This task is usually performed by a classifier or model based on features present in this data. Some examples of classification tasks are: character recognition \cite{mnist}, spam detection \cite{spambase} and face recognition \cite{face_recognition_survey}. These tasks are generally very daunting for humans causing a great interest in improving the accuracies of these classifiers.\\

The first classifiers made use of algorithms such as nearest neighbors \cite{nearest_neighor}, decision trees \cite{decision_tree} and \glspl{svm} \cite{svm}. More recent classifiers are based on neural networks, which in turn are inspired by the human brain. These networks come in a great variety of forms and show near-human performance on some very specific tasks \cite{alpha_go_google}. This improvement in performance caused even more interest in the applications of \gls{ai} and \gls{ml}. Forbes \cite{forbes} recently predicted that in ten years \gls{ai} will be present in all areas of our society. The next decade is deemed as the \textit{"most promising era in technology innovation"} \cite{forbes}.\\

The prevalence of \gls{ml} in the near future does not only bring opportunities. It also poses some threats that may not be obvious at first glance. A spam email passing through an \gls{ai}-based spam filter might not be the end of the world, while mispredicting some characters from an image can have a bigger impact depending on the context. False predictions in the medical domain might be even more severe, but they could be overturned by a doctor. Self-driving cars not detecting traffic signs could put several lives at risk. Depending on the situation in which the classifier is deployed, some threats can be more dangerous than others.\\

All previous examples of threats are due to the inaccuracies of the classifier itself. There are also some threats caused by the inherent nature of neural networks. In 2014, it was shown that neural networks are prone to adversarial examples \cite{szegedy2014intriguing}. An adversarial example consists of a correctly classified data instance and a small perturbation. This perturbation causes the neural network to misclassify the instance. This allows malicious users to create images that are seemingly identical, while being classified differently. This poses an even greater threat to the security of \gls{ai} than the inaccuracies of the classifiers.\\

The process of creating an adversarial example is called an adversarial attack. Different types of attacks exist depending on the information available to the attacker. White-box attacks require complete knowledge of the classifier under attack, while black-box attacks only require the output(s) of the model. All attacks require sending one or more queries to the classifier.\\

Ever since the discovery of adversarial examples, an arms race between adversarial attack and defense researchers has been taken place. In the current state of affairs in the adversarial \gls{ml} domain, the stateful defensive mechanism \cite{chen_stateful_2019} is considered the state-of-the-art. Unlike previous defensive schemes, this scheme holds state of previously submitted queries. This allows the scheme to make a decision based on a series of queries instead of a single query.\\

The stateful defense mechanism makes the assumption that there is no collaboration between different users of the model and that every submitted query can be traced back to the submitting user. This is not necessarily the case in real scenarios, since users are free to work together. It is even possible for one user to set up multiple accounts and essentially collaborate with itself. This work aims to exploit the assumption made by the defensive mechanism and create adversarial examples while triggering as few detections as possible.\\

This goal is achieved by first altering an existing adversarial attack using \gls{pso} in order to bypass the stateful defense mechanism. \gls{pso} is an optimization framework inspired by the flocking of birds. Afterwards multiple ideas are explored in an effort to improve the efficiency or evasiveness of the altered attack. Some ideas are specific for the created attack, while others are generally applicable to all other adversarial attacks.\\

The work is structured as follows: Chapter \ref{chap:background} discusses the necessary background needed in order to comprehend the rest of the work. Chapter \ref{chap:related_work} gives an overview of work related to this subject. Some specific adversarial attacks are discussed as is the stateful defense mechanism. Chapter \ref{chap:approach} introduces the main ideas used in the development of the altered attack. This chapter also poses some interesting research questions. Chapter \ref{chap:evaluation} proposes a novel attack and iteratively refines this attack in order to increase the efficiency or evasiveness. At the end of every section of this chapter, some conclusions are drawn and discussed. Chapter \ref{chap:discussion} contains a discussion about the work as a whole. Some final remarks are given as well as some pointers for future work. Finally, Chapter \ref{chap:conclusion} concludes this work by answering the research questions posed in Chapter \ref{chap:approach}. \\ 
\chapter{Background}
\section{Neural networks}
Ever since the invention of computer systems, it has always been a goal of scientists and engineers to create \gls{ai}. Current state of the art approaches are mimicking the human brain, more specifically the neurons inside the brain. Already in the fifties, Rosenblatt introduced his perceptron \cite{rosenblatt_perceptron_1958}. The perceptron is a single neuron able to learn linearly separable patterns. It does so by finding a hyperplane that separates the two classes. This hyperplane is called the decision surface or decision boundary. The concept of linear separability is explained in Figure \ref{fig:linear_separability} in two dimensions. \\

Unfortunately not all patterns are linearly separable. To overcome this problem, the neurons can be layered, creating an \gls{ann} in the process. Layering neurons sequentially is essentially a linear combination of neurons. This in itself does not create non-linear decision surfaces. Non-linear activation functions are added for the \gls{ann} to be able to learn more complex decision boundaries. Some commonly used activation functions are \gls{relu} \cite{relu}, Heaviside step function and softmax (or sigmoid when used on scalars). In Figure \ref{fig:activation_functions} the plots of the activation functions can be found. 



\begin{figure}
\tikzset{every picture/.style={line width=0.75pt}} %set default line width to 0.75pt        
\centering
\begin{tikzpicture}[x=0.75pt,y=0.75pt,yscale=-1,xscale=1]
%uncomment if require: \path (0,300); %set diagram left start at 0, and has height of 300

%Straight Lines [id:da34928766651876697] 
\draw [line width=1.5]    (40,220) -- (190,220) ;
%Straight Lines [id:da5068123062544778] 
\draw [line width=1.5]    (40,120) -- (40,220) ;
%Straight Lines [id:da02959929373017789] 
\draw [color={rgb, 255:red, 247; green, 0; blue, 0 }  ,draw opacity=1 ]   (71,133) -- (81,133) ;
%Straight Lines [id:da023175201693992564] 
\draw [color={rgb, 255:red, 65; green, 117; blue, 5 }  ,draw opacity=1 ]   (132,166) -- (142,166) ;
%Straight Lines [id:da12599178724449778] 
\draw [color={rgb, 255:red, 65; green, 117; blue, 5 }  ,draw opacity=1 ]   (137,161) -- (137,171) ;

%Straight Lines [id:da014209875608111044] 
\draw [color={rgb, 255:red, 65; green, 117; blue, 5 }  ,draw opacity=1 ]   (95,185) -- (105,185) ;
%Straight Lines [id:da8046955093022794] 
\draw [color={rgb, 255:red, 65; green, 117; blue, 5 }  ,draw opacity=1 ]   (100,180) -- (100,190) ;

%Straight Lines [id:da8257858393303268] 
\draw [color={rgb, 255:red, 65; green, 117; blue, 5 }  ,draw opacity=1 ]   (151,177) -- (161,177) ;
%Straight Lines [id:da9904899777355851] 
\draw [color={rgb, 255:red, 65; green, 117; blue, 5 }  ,draw opacity=1 ]   (156,172) -- (156,182) ;

%Straight Lines [id:da4558392192767111] 
\draw [color={rgb, 255:red, 65; green, 117; blue, 5 }  ,draw opacity=1 ]   (132,189) -- (142,189) ;
%Straight Lines [id:da6277106773918533] 
\draw [color={rgb, 255:red, 65; green, 117; blue, 5 }  ,draw opacity=1 ]   (137,184) -- (137,194) ;

%Straight Lines [id:da6088809310233947] 
\draw [color={rgb, 255:red, 65; green, 117; blue, 5 }  ,draw opacity=1 ]   (131,142) -- (141,142) ;
%Straight Lines [id:da9717175839575936] 
\draw [color={rgb, 255:red, 65; green, 117; blue, 5 }  ,draw opacity=1 ]   (136,137) -- (136,147) ;

%Straight Lines [id:da05579332834233042] 
\draw [color={rgb, 255:red, 247; green, 0; blue, 0 }  ,draw opacity=1 ]   (84,146) -- (94,146) ;
%Straight Lines [id:da9339907902154778] 
\draw [color={rgb, 255:red, 247; green, 0; blue, 0 }  ,draw opacity=1 ]   (64,181) -- (74,181) ;
%Straight Lines [id:da801343700133325] 
\draw [color={rgb, 255:red, 247; green, 0; blue, 0 }  ,draw opacity=1 ]   (103,127) -- (113,127) ;
%Straight Lines [id:da6207217857469742] 
\draw [color={rgb, 255:red, 247; green, 0; blue, 0 }  ,draw opacity=1 ]   (58,157) -- (68,157) ;

%Straight Lines [id:da9774802467048496] 
\draw [line width=1.5]    (238,221) -- (388,221) ;
%Straight Lines [id:da8926029470752344] 
\draw [line width=1.5]    (238,121) -- (238,221) ;
%Straight Lines [id:da391932663930169] 
\draw [color={rgb, 255:red, 247; green, 0; blue, 0 }  ,draw opacity=1 ]   (360,200) -- (370,200) ;
%Straight Lines [id:da45698367355392255] 
\draw [color={rgb, 255:red, 65; green, 117; blue, 5 }  ,draw opacity=1 ]   (330,167) -- (340,167) ;
%Straight Lines [id:da8509197980874896] 
\draw [color={rgb, 255:red, 65; green, 117; blue, 5 }  ,draw opacity=1 ]   (335,162) -- (335,172) ;

%Straight Lines [id:da46487388742354385] 
\draw [color={rgb, 255:red, 65; green, 117; blue, 5 }  ,draw opacity=1 ]   (293,186) -- (303,186) ;
%Straight Lines [id:da6402947556346299] 
\draw [color={rgb, 255:red, 65; green, 117; blue, 5 }  ,draw opacity=1 ]   (298,181) -- (298,191) ;

%Straight Lines [id:da371430190479477] 
\draw [color={rgb, 255:red, 65; green, 117; blue, 5 }  ,draw opacity=1 ]   (349,178) -- (359,178) ;
%Straight Lines [id:da029152498087926304] 
\draw [color={rgb, 255:red, 65; green, 117; blue, 5 }  ,draw opacity=1 ]   (354,173) -- (354,183) ;

%Straight Lines [id:da7641303373625803] 
\draw [color={rgb, 255:red, 65; green, 117; blue, 5 }  ,draw opacity=1 ]   (330,190) -- (340,190) ;
%Straight Lines [id:da8503441823041022] 
\draw [color={rgb, 255:red, 65; green, 117; blue, 5 }  ,draw opacity=1 ]   (335,185) -- (335,195) ;

%Straight Lines [id:da14090483844456858] 
\draw [color={rgb, 255:red, 65; green, 117; blue, 5 }  ,draw opacity=1 ]   (329,143) -- (339,143) ;
%Straight Lines [id:da10211881619757635] 
\draw [color={rgb, 255:red, 65; green, 117; blue, 5 }  ,draw opacity=1 ]   (334,138) -- (334,148) ;

%Straight Lines [id:da8074255152632699] 
\draw [color={rgb, 255:red, 247; green, 0; blue, 0 }  ,draw opacity=1 ]   (296,208) -- (306,208) ;
%Straight Lines [id:da01705778265856206] 
\draw [color={rgb, 255:red, 247; green, 0; blue, 0 }  ,draw opacity=1 ]   (363,153) -- (373,153) ;
%Straight Lines [id:da4845113413868791] 
\draw [color={rgb, 255:red, 247; green, 0; blue, 0 }  ,draw opacity=1 ]   (283,133) -- (293,133) ;
%Straight Lines [id:da861487465385409] 
\draw [color={rgb, 255:red, 247; green, 0; blue, 0 }  ,draw opacity=1 ]   (256,158) -- (266,158) ;
%Straight Lines [id:da9913185077125697] 
\draw [color={rgb, 255:red, 247; green, 0; blue, 0 }  ,draw opacity=1 ]   (321,120) -- (331,120) ;
%Straight Lines [id:da016655837516902583] 
\draw [color={rgb, 255:red, 65; green, 117; blue, 5 }  ,draw opacity=1 ]   (301,161) -- (311,161) ;
%Straight Lines [id:da48554389877316395] 
\draw [color={rgb, 255:red, 65; green, 117; blue, 5 }  ,draw opacity=1 ]   (306,156) -- (306,166) ;


%Straight Lines [id:da7796305041163321] 
\draw [color={rgb, 255:red, 74; green, 144; blue, 226 }  ,draw opacity=1 ][line width=2.25]    (139,110.5) -- (67,207.5) ;

% Text Node
\draw (319,222) node [anchor=north west][inner sep=0.75pt]   [align=left] {Feature 1};
% Text Node
\draw (219,188) node [anchor=north west][inner sep=0.75pt]  [rotate=-270] [align=left] {Feature 2};
% Text Node
\draw (121,221) node [anchor=north west][inner sep=0.75pt]   [align=left] {Feature 1};
% Text Node
\draw (21,187) node [anchor=north west][inner sep=0.75pt]  [rotate=-270] [align=left] {Feature 2};


\end{tikzpicture}
\caption[Linear separability]{Linearly separable classes on the left and non-linearly separable classes on the right. Two classes are linearly separable if there exists a hyperplane for which all examples of one class are on the same side of this hyperplane, whilst all examples of the other class are on the other side of the hyperplane. In two dimensions, the hyperplane is a straight line.}
\label{fig:linear_separability}
\end{figure}

\begin{figure}
\centering
\begin{tikzpicture}
\begin{axis}[width=0.30\textwidth,
height=4cm,
axis lines=middle,
xlabel=$x$,
ylabel=$y$,
xmin=-1,
xmax=1,
ymin=0,
ymax=1,
xtick={-1,1},
ytick={0,0.5,1},
axis line style={-latex},
ticklabel style={font=\tiny,fill=white},
]
\addplot[ultra thick, color=red]{max(0,x)};
\end{axis}
\end{tikzpicture}
\begin{tikzpicture}
\begin{axis}[width=0.30\textwidth,
height=4cm,
axis lines=middle,
xlabel=$x$,
ylabel=$y$,
xmin=-1,
xmax=1,
ymin=0,
ymax=1,
xtick={-1,1},
ytick={0,0.5,1},
axis line style={-latex},
ticklabel style={font=\tiny,fill=white},
]
\addplot+[const plot, no marks, ultra thick, color=red] coordinates {(-10,0) (0,1) (10, 1)};
\end{axis}
\end{tikzpicture}
\begin{tikzpicture}
\begin{axis}[width=0.30\textwidth,
height=4cm,
axis lines=middle,
xlabel=$x$,
ylabel=$y$,
xmin=-5,
xmax=5,
ymin=0,
ymax=1,
xtick={-5,5},
ytick={0,0.5,1},
axis line style={-latex},
ticklabel style={font=\tiny,fill=white},
]
\addplot[ultra thick, color=red]{exp(x) / (1 + exp(x))};
\end{axis}
\end{tikzpicture}
\caption[Activiation functions]{Plots of different activation functions. From left to right: \gls{relu}, Heaviside step and sigmoid.}
\label{fig:activation_functions}
\end{figure}


\section{Adversarial attacks}

\section{Particle swarm optimization}





\bibliographystyle{unsrt}
\bibliography{bibliography}
\end{document}